\thispagestyle{empty}   % 定义起始页的页眉页脚格式为 empty —— 空,也就没有页眉页脚

\begin{center}
    \textbf{\fontsize{20}{1.5}基于客流预测的地铁发车间隔优化模型}
    \newline
    \textbf{\fontsize{12}{1.5}A Subway Departure Interval Optimization Model Based on Passenger Flow Prediction}
    \newline
    \textbf{\fontsize{12}{1.5}EFLKumo and Other Authors}

%   \textbf{摘 要}
\end{center}





% ==================================================
%
%   摘要
%
% --------------------------------------------------


\section*{摘要}

本研究致力于解决城市地铁运营中的发车规划问题,提高地铁运营效率。针对传统策略中固定发车间隔难以应对客流时空分布不均的挑战,
本文提出了一种基于客流预测的发车间隔优化模型,该模型通过分析客流量随时间变化的通用函数,充分考虑乘客体验、运营成本等多重因素,能够动态调整发车间隔,满足不同时段客流需求。

同时,本研究通过分析多个站点的不同流量特征,得到通用数学模型,站点只需要通过数据采样或历史数据即可得到适用于本站点的个性化模型。

本研究为城市交通轨道运营优化提供了理论参考。\newline
\newline
\textbf{关键词}:多目标规划 \quad 动态客流预测 \quad 发车间隔优化 \quad 地铁运营