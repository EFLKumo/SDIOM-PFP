\thispagestyle{empty}   % 定义起始页的页眉页脚格式为 empty —— 空,也就没有页眉页脚

\begin{center}
    \textbf{\fontsize{20}{1.5}基于客流预测的地铁发车间隔优化模型}
    \newline
    \textbf{\fontsize{12}{1.5}A Subway Departure Interval Optimization Model Based on Passenger Flow Prediction}
    \newline
    \textbf{\fontsize{12}{1.5}EFLKumo and Other Authors}

%   \textbf{摘 要}
\end{center}





% ==================================================
%
%   摘要
%
% --------------------------------------------------


\section*{摘要}

为提升乘客体验、优化运营成本,城市轨道交通系统的发车间隔需要时刻变化。就目前各城市轨道交通中站内拥挤、运送空车等现象,
针对传统策略中人工排班难以科学应对客流时空分布不均、多方面因素权衡不足的挑战,本研究提出了一种基于客流预测的发车间隔优化模型。
该模型通过分析站点客流量随时间的变化,充分考虑乘客体验、运营成本等多重因素,能够动态调整发车间隔,满足不同时段客流需求。

同时,本研究提取了建立该模型的通用步骤,并提供了相关代码。运营方可以通过历史数据和当天数据的实时采样,得到适用于特定站点的个性化模型。

本研究初步解决城市地铁运营中的发车规划问题,有利于提高轨道交通运营效率、乘客满意度,为地铁科学调度提供了量化决策支持。\newline
\newline
\textbf{关键词}:多目标规划 \quad 客流预测 \quad 发车时刻表优化 \quad 地铁运营