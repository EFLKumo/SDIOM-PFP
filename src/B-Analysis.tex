% 设置页码计数器为 1 (也就是当前页面为第一页)
\setcounter{page}{1}

\mcmSection{问题重述}

\mcmSubsection{问题背景}

现代城市轨道交通系统中,地铁发车间隔的优化直接影响乘客出行体验和运营经济效益。根据北京交通发展研究院数据 \cite{BeijingTransportDevReport2024},高峰时段地铁站台进出站量可占全天进出站量的 50\%以上。
固定发车间隔模式存在高峰时段运力不足、平峰时段资源浪费的矛盾。以往研究重视完整线路的发车时刻表优化,缺乏对单个站点基于客流实际情况的考虑。

本研究针对各站点客流数据,构建客流预测模型,通过遗传算法提出发车建议,在保证基础运力的前提下,平衡乘客等待成本与运营成本最小化双重目标。

\mcmSubsection{具体问题重述}

我们已知地铁入站人数随时间的变化函数,需要设计一个模型指导地铁运营方设定列车发车间隔,使得“站内总人数”或“乘客平均等待时间”等标准维持在合理范围内,同时充分考虑运营成本,最终实现发车时刻优化。

% ==================================================
% @brief    问题分析
% ==================================================

\mcmSection{问题分析}

乘客入站人数函数 $P(t)$ 是规划发车间隔的核心输入信息。$P(t)$ 的波动性是我们需要重点考虑的,因为在高峰期和低峰期,乘客的需求差异巨大,固定的发车间隔显然无法满足所有情况。
同时,列车从始发站开到本站点需要一定的时间,这就要求 $P(t)$ 是一个能够预测客流的函数。

运营成本主要与发车频率相关。更频繁的发车意味着更高的能源消耗、车辆维护成本和人力成本。在满足乘客需求的前提下,尽可能降低运营成本也是地铁运营方的重要目标。

为了简化问题,我们只使用各个站点 0:00~12:00 的上半天数据,这样就只用考虑单高峰。实际上,在解决该问题时,上半天与下半天是完全可分离的。

同时,我们建立模型时需要考虑可调整性,地铁运营方可以使用历史甚至当天的实时数据对模型进行修正,以实现最佳规划。

综上所述,我们需要设计可预测客流、方便修正的函数 $P(t)$,并通过遗传学算法基于 $P(t)$ 给出发车建议。

% ==================================================
% @brief    模型假设
% ==================================================

\mcmSection{模型假设}

\begin{enumerate}
    \item 假设一:乘客到达车站后,会立即进入候车状态,并且会选择乘坐最早到达的列车。乘客不会因为等待时间过长而放弃乘坐地铁,也不会有其他复杂的行为模式影响客流的疏散。
    \item 假设二:地铁列车的载客能力是固定的,并且已知。每列列车能够容纳的乘客数量是有限的,这个限制会影响到发车间隔的设定,以避免列车超载。
    \item 假设三:列车在站点的停靠时间是恒定的,并且相对于发车间隔而言可以忽略不计,或者已经包含在发车间隔的计算中。我们主要关注的是列车出发的时间间隔,而非在站点的具体停靠时长。
    \item 假设四:运营成本主要与发车频率成正比关系。发车频率越高,运营成本越高。我们可以用一个简化的成本函数来描述运营成本与发车频率之间的关系,例如线性关系或者其他合理的函数形式。
    \item 假设五:暂时忽略一些次要因素,例如不同线路之间的换乘时间、突发事件对客流的影响等。
\end{enumerate}

% ==================================================
% @brief    符号说明及名称定义
% ==================================================

\mcmSection{符号说明及名称定义}

\begin{table}[ht]
    % 表格居中
    \centering

    % 调整行距
    \renewcommand\arraystretch{1.5}
    
    % 放缩表格
    \scalebox{1.2}{

        \begin{tabular}{cc}
        \hline
        \textbf{\fontsize{13}{1.5}{符号}}   & \textbf{\fontsize{13}{1.5}{意义}}                       \\ \hline

        $t$                                 & 时间变量,以分钟为单位     \\
        $E_i(t)$                            & 单站点 0:00~12:00 的每分钟进站人数 \\
        $\text{LOESS}(f, frac)$                    & 对 $f$ 进行 LOESS 处理,使用 $frac$ 作为参数 \\
        $\text{DTW\_Align}(f,g_i,g_{ref})$  & 以 $g_i$ 向 $g_{ref}$ 对齐作为路径参考,对 $f$ 进行对齐 \\
        $\gamma$                            & 乘客等待时间成本系数 \\
        $\eta$                              & 发车固定成本系数 \\ \hline
        \end{tabular}
    
    }
\end{table}
